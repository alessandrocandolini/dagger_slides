\PassOptionsToPackage{table}{xcolor}
\documentclass[10pt]{beamer}
\usepackage[english]{babel}

\usetheme{metropolis}
\usepackage{smartdiagram}
\usepackage{listings}
\usepackage{booktabs}
\usepackage[scale=2]{ccicons}%creative commons
\setbeamercovered{transparent}%invisible by default
\usepackage{array}
\newcolumntype{L}[1]{>{\raggedright\let\newline\\\arraybackslash\hspace{0pt}}m{#1}}
\newcolumntype{C}[1]{>{\centering\let\newline\\\arraybackslash\hspace{0pt}}m{#1}}
\newcolumntype{R}[1]{>{\raggedleft\let\newline\\\arraybackslash\hspace{0pt}}m{#1}}

\usepackage{pgfplots}
\usepgfplotslibrary{dateplot}
\usepackage{tikz}
\usepackage{tikz-uml}
\usetikzlibrary{positioning,chains,fit,shapes,calc}

\newcommand{\mycomment}[1]{}
\usepackage{fancyvrb}
% *****************************************************************************
% Matematica 
% *****************************************************************************

%\usepackage{amssymb}
%\usepackage{mathtools}                    % Add support for cramped,
					  
%\usepackage[euler]{flexisym}
%\usepackage{breqn}                        % Breqn
%\makeatletter
%   \def\eqnumsize{\normalfont \Tf@font}      % Add support to Minion Pro
%\makeatother
%\setkeys{breqn}{labelprefix={eq:}}


%\usepackage{asymptote}
%\usepackage[loop, controls]{animate}

\graphicspath{{./}, {./Images/}}

\lstdefinelanguage{Kotlin}{
  keywords={package, as, typealias, this, super, val, var, fun, for, null, true, false, is, in, throw, return, break, continue, object, if, try, else, while, do, when, yield, typeof, yield, typeof, class, interface, enum, object, override, public, private, get, set, import, abstract, },
  keywordstyle=\color{blue}\bfseries,
  ndkeywords={@Deprecated, Int, Integer, Float, Double, String, Runnable, dynamic},
  ndkeywordstyle=\color{red}\bfseries,
  emph={println, return@, forEach,},
  emphstyle={\color{red}},
  identifierstyle=\color{black},
  sensitive=true,
  commentstyle=\color{gray}\ttfamily,
  comment=[l]{//},
  morecomment=[s]{/*}{*/},
  stringstyle=\color{gray}\ttfamily,
  morestring=[b]",
  morestring=[s]{"""*}{*"""},
}

\providecommand{\ie}{i.\,e.}
\providecommand{\Ie}{I.\,e.}
\providecommand{\eg}{e.\,g.}
\providecommand{\Eg}{E.\,g.} 

\metroset{block=fill}
\metroset{titleformat frame=smallcaps}

\title{Dependency injection made easy with Dagger2}
%\subtitle{and how it can help us building better reactive code}

\date{\today}
\author[A. Candolini]{Alessandro Candolini}
%\institute{Department of Physics, University of Trieste}
% \titlegraphic{\hfill\includegraphics[height=1.5cm]{logo/logo}}

\begin{document}

\maketitle

\begin{frame}{Agenda}
  \setbeamertemplate{section in toc}[sections numbered]
  \tableofcontents[hideallsubsections]
\end{frame}

\section{Dependency injection principles}
\begin{frame}[fragile]
	\frametitle{What is a dependency?}
		\begin{figure}
			\centering
\begin{tikzpicture}
\umlsimpleclass [minimum height=15ex,width=15ex] {A}
	\umlsimpleclass[x=4,width=5ex]{B}
\umlunicompo[geometry=|-|]{A}{B}
\end{tikzpicture}
		\end{figure}
\end{frame}
\begin{frame}[fragile]
%\begin{lstlisting}[language=java,basicstyle=\ttfamily,keywordstyle=\color{red}]
\begin{lstlisting}[language=Kotlin, basicstyle=\ttfamily]
/** Class A */
class A {
    // ....
    fun doSomething() {
        b.log("text")
    }
}

/** Class B (dependency) */
class B {
    fun log(text : String) {
    }
}
\end{lstlisting} 
\end{frame}
	\begin{frame}[fragile]
\begin{lstlisting}[language=Kotlin, basicstyle=\ttfamily]
// Option 1 - static methods 

class A {
    fun doSomething() {
        B.log("text") // <- static method
    }
}

class B {
    companion object {
        fun log(text: String) {
        }
    }
}
\end{lstlisting} 
	\end{frame}

	\begin{frame}
		Examples:
		\begin{itemize}
			\item Helper  classes
			\item Utils  classes
			\item Manager classes, etc\ldots
		\end{itemize}
	\end{frame}

	\begin{frame}[fragile]
		Drawbacks:
		\begin{itemize}
			\item $A$ not testable in isolation 
			\item $A$ \emph{tightly coupled} to $B$
			\item Lack of encapsulation  (backdoor) 
			\item \emph{Hidden} dependency 
		\end{itemize}
	\end{frame}
	\begin{frame}[fragile]
		More Examples:
		\begin{itemize}
			\item \verb|Application.getStaticContext()\|
			\item migrating one class leads to migrate 100 classes\ldots 
		\end{itemize}
	\end{frame}
\begin{frame}[fragile]
\begin{lstlisting}[language=Kotlin, basicstyle=\ttfamily]
// Option 2 - singletons

class A {
    fun doSomething() {
	B.log("text") // <- singleton
    }
}

object B {
    fun log(text: String) {
    }
}
\end{lstlisting} 
\end{frame}
\begin{frame}[fragile]
\begin{lstlisting}[language=Kotlin, basicstyle=\ttfamily]
// Option 3 - composition 

class A {
    private val b : B = B() // <-- instantiate
    
    fun doSomething() {
        b.log("text")
    }
}

class B {
    fun log(text: String) {
    }
}
\end{lstlisting} 
\end{frame}
	\begin{frame}[fragile]
		\frametitle{Composition}
		\begin{figure}
			\centering
\begin{tikzpicture}
	\umlclass{A}{}{+ doSomething() : void}
	\umlclass [x=5, y=0] {B}{}{+ log(text : String) : void}
\umlunicompo{A}{B}
\end{tikzpicture}
		\end{figure}
%\begin{tikzpicture}
%\umlsimpleclass [x=0,width=15ex] {A}
%\umlsimpleclass [ x=10ex, y=0] {B}
%\umlunicompo[geometry=|-|]{A}{B}
%\end{tikzpicture}
	\end{frame}

	\section{Dagger2}
	\section{Dagger2 Android }
\plain{Questions?}

%\begin{frame}[allowframebreaks] {References}
% \bibliography{demo}
% \bibliographystyle{abbrv}
%\end{frame}

\end{document}
